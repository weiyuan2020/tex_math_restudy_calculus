\chapter{重修微积分4——距离}

上一篇从很抽象的角度,介绍了能够支持收敛极限的数学空间。用拓扑空间的定义和例子,从高处俯瞰你学过的初等微积分,把散落的知识,用概念间的联系组织起来,让你看到这些基本概念构筑成一个概括的,无穷空间模糊的图像。

学习没有捷径,唯常习练才能走通。旁人传的只是心法体会。如果没有过去学的微积分知识作为基础,这系列的内容即便记住了,也浮在云中,不过是观光游览。只有用学过的定理公式作为理解概念的桩脚,约束散漫的联想,抽象概念的想象才能落到实处。只有将文中定义,示例和关联的陈述,逐个用逻辑走通,勾画出来的骨架上才有色彩,才有直观想象。正确的想象和浮想的区别,前者可以作为证明的思路,其间的推理能用数学语言来表达和证明。而浮想除了娱乐,推论犹如算卦,不能明确也就没有实用的价值。

如果你已经消化了前几篇,这里将介绍具有更为直观的空间,让你头脑中图像更加鲜明。

很抽象的拓扑空间,缺乏足够的性能可直接用于物理和工程。这一篇和下面一篇,介绍在理论物理和实变泛函课中常见到的,拥有更多性质的拓扑空间。通过它们的联系,揭示相关的数学概念。读者试着从定义来验证所举的例子和关联陈述,用逻辑走通来消化概念,细化前篇建立起来的直观想象。

在集合上的两个点间,如果有了``距离''这个度量,将是应用者最易于想象远近相邻概念的拓扑。这时趋近、收敛和极限,就非常直观了。所以它在物理和数学中有着广泛的应用。这样的拓扑空间叫做距离空间(Metric space),有时也译为``度量空间''。

在集合上怎么抽象化``距离'',让它有最大外延又符合已有的直观想象?首先,它是个两点间非负实数值的度量,0值意味着相等,非0视为不同,这度量对两点是对称的,三点相互之间犹如三角形的三边长度关系。

\kaishu
\setlength{\leftskip}{1em}

在集合M上,距离定义为一个二元实数值函数 $ \rho : M\times M \rightarrow\mathbb{R} $ ,对M上任意的x, y, z有下列非负,相等,对称,三角不等式的性质:

$ \rho(x,y)\geq 0 , \rho(x,y)=0 \Leftrightarrow
x=y , \rho(x,y)=\rho(y,x), \rho(x,y)+\rho(y,z)\geq \rho(x,z) $
\songti
\setlength{\leftskip}{0em}

对于任何实数 $ r>0,x_0\in M $,集合$ \{x\in M \;|\; \rho(x,x_0)<r\} $定义了一个以$ x_0 $为中心,$ r $为半径的\textbf{开球}。以它们为开集,生成了M上的拓扑,它构成的拓扑空间称为\textbf{距离空间},记为$ (M,\rho) $。实际上这空间的开集都能用开球的并来构成,这样的一组开集(在距离空间是开球)称为\textbf{拓扑基}。具有可数拓扑基的空间,称为\textbf{第二可数}的。第二可数的空间都是第一可数的。想象一下实数中以有理数为边界值的开区间,它们的各种并集构造出实数上所有的开集。同理可证$ \mathbb{R}^n $空间(在通常的拓扑下),都是第二可数的空间。一般距离空间不一定是第二可数的。但都是第一可数的。

可以证明,距离空间中点的邻域,都包含有以这点为中心足够小半径的开球,所以一般拓扑空间中用邻域表达收敛的定义,可以用距离表达如下:

\kaishu
\setlength{\leftskip}{1em}

记无穷序列 $ x_1,x_2,x_3,\cdots $ 为$ (x_n) $,集合$ X $上的无穷序列$ (x_n) $收敛于$ a $,即 $ \lim_{n \rightarrow \infty} a_n $$= a $ 或者 $ a_n \rightarrow a , \mbox{ } (n \rightarrow \infty) $,意思是:

\[\forall \varepsilon > 0, \;\exists N \in \mathbb{N},\;\forall n ( n>N \Rightarrow \rho (a_n,a) < \varepsilon )\]

\songti
\setlength{\leftskip}{0em}

距离空间的邻域必定包含有半径为有理数的小球,所以它是第一可数的,集合的聚点都有集合中的一个序列来趋近它。

在实数上二元函数$ d(x,y)=|x-y| $,它满足距离的定义,所以$ (\mathbb{R}, d) $是个距离空间。以此代入上面公式,就是用$ \varepsilon - N $ 语言表达的数列收敛的定义。

在同一个集合上,可以有多种方法来定义距离。例如,对n维实数空间上两个点,$ x,y \in \mathbb{R}^n $
,不难验证函数$ d, d_1, d_2 $ 都可以定义为$ \mathbb{R}^n $ 上的距离。

\[d(x,y)=\sqrt{\sum_{i=1}^n (x_i  −y_i)^2}, d_1(x,y)=\sum_{i=1}^{n}|x_i − y_i|, d_2(x,y)=\max_i{|x_i−y_i|}\]

特别地,距离空间$ (\mathbb{R}^n,d) $ 称为$ n $维欧几里德空间。上面谈到的开球概念,是逻辑上离中心距离小于一个数的点集合,你可以把它想象成在三维各向同性空间的球体。对 $ d, d_1, d_2 $ 不同距离定义的开球,如在变形的空间,具有不同的几何形状。但相互间,放大一个便能将另一包住,所以用不同距离定义出的收敛都是一样的(等价的),它们构成的空间并没有本质的区别。下面会看到,对一般的距离空间并非都是如此。

同样地,可以定义\textbf{柯西列}$ (x_n) $ :

\[\forall \varepsilon >0, \; \exists N\in \mathbb{N}, \;\forall n \forall m(m,n>N \rightarrow \rho(a_n,a_m)<\varepsilon)\]

收敛的序列都是柯西列,\textbf{完备的距离空间}中柯西列都收敛。欧几里德空间是完备的。

到目前为止,我们可以用实数上收敛极限的图像,类比地想象距离空间相应的概念。距离空间的收敛、极限和柯西列的定义类同于实数空间。点的序列类同于数列,开球相似于开区间,开集、闭集、闭包和覆盖定义性质都一样,我们可以借用实数空间的直观来想象距离空间。下面介绍距离空间的一些重要性质,来修正你过分的联想。

并非所有距离空间都有柯西列。

\kaishu
\setlength{\leftskip}{1em}

例4.1:对集合X上任意两个不同点x,y 定义函数d(x,x)=0,d(x,y)=1。显然它符合距离的定义。(X,d)是个距离空间,叫``离散空间'',具有离散拓扑,没有柯西列。

\songti
\setlength{\leftskip}{0em}

对于相同的基础集合,在上面定义不同距离,并非空间上收敛都是等价的。

\kaishu
\setlength{\leftskip}{1em}

例4.2:记C[a,b]为实数闭区间[a,b]到R所有连续函数的集合,对$ f,g\in C[a,b] $可以定义:

\[ \displaystyle d_1(f,g) = \int_{a}^{b}|f(t)-g(t)| \]

\[ \displaystyle d_\infty(f,g) = \max_{a\leq t \leq b} |f(t)-g(t)|\]

\songti
\setlength{\leftskip}{0em}

$ (C[a,b],d_1) $ 和$ (C[a,b],d_\infty) $ 都是距离空间,因为有 $\displaystyle  d1(f,g)\leq (b−a)d_\infty(f,g) $ ,如果函数序列$ (f_n) $ 按距离$ d_\infty $收敛的(一致收敛),则按距离$ d_1 $也必定收敛(积分平均收敛),反之则不然。

\kaishu
\setlength{\leftskip}{1em}

例4.3:闭区间$ [0,1] $上连续函数集合$ C[0,1], f_n(t)=t^n $看作是下标为n的点,序列 $ (f_n) $(或直接写成$ (t^n) $ ),按距离$ d_1 $ 是柯西列,按$ d_\infty $ 则不是,在这两个距离空间上它都不收敛。但在$ [0,1] $上可积函数集合里,依$ d_1 $距离,($ f_n(t) $)收敛到一个不连续函数 $ f(t): f(1)=1,0\leq t<1,f(t)=0 $,这个极限$ f(t) $不在$ C[0,1] $ 里。

\songti
\setlength{\leftskip}{0em}

如果距离空间X中,任何柯西列都收敛到X中的一点,则称它是完备的。并非所有距离空间都是完备的。

\kaishu
\setlength{\leftskip}{1em}

例4.4:$ (C[a,b],d_\infty) $是完备的(连续函数序列一致收敛的极限,是连续函数)。$ (C[a,b],$$d_1) $ 不是完备的(连续函数序列按积分平均收敛的极限,不一定是连续函数)。

例4.5:开区间$ (0, 1) $作为实数拓扑子空间不是完备的。柯西列$ (1/n) $在那里没极限。

\songti
\setlength{\leftskip}{0em}

实数空间是完备的,在有界和闭时就有了很好的收敛性质。对一般完备的距离空间,有界则要换成更强的完全有界的条件。下面介绍距离空间中这些概念和关系。

如果集合M中任意两点的距离,都小于一个正数,则称集合M是 \textbf{有界的} (Bounded);如果对于任何 $ r> 0 $,都有限多个,半径为 r 的开球覆盖 M,则称M是\textbf{完全有界}(Totally   Bounded);如果M的开覆盖都有有限子覆盖,则称M是\textbf{紧的}或\textbf{紧致的}(Compact);如果任一无穷序列,都有一个收敛的子列,则称M是\textbf{列紧的}(Sequentially Compact)。

实数空间和欧几里德空间是距离空间。它们都是完备的,在那里有界的集合是完全有界的,对它们有列紧性定理:有界数列,都有一个收敛的子列。但这定理对一般距离空间不成立。

\kaishu
\setlength{\leftskip}{1em}

例4.6:无穷序列($ t_n $)在$ (C[a,b],d_1) $和$ (C[a,b],d_\infty) $距离空间都是有界,却都没有收敛的子列。前者是不收敛的柯西列,后者不是柯西列,也没有柯西子列。

\songti
\setlength{\leftskip}{0em}

在一般的距离空间,一个无穷序列,未必有柯西子列,但如果它是完全有界的,则有柯西子列;如果它还是完备的,则有收敛的子列,即是列紧的。

\kaishu

\textbf{这三个性质在距离空间中等价:完全有界且完备的,列紧的,紧的。}

\songti

距离空间的子集如果是完全有界的,则它也是有界的,反之则不然。

\kaishu
\setlength{\leftskip}{1em}

例4.7:离散空间是有界的,但无穷集合的离散空间不是完全有界的。

\songti
\setlength{\leftskip}{0em}

距离空间如果是完备的,它的闭集形成的子空间也是完备的。紧集是完全有界和闭的。实数的有界闭区间,欧几里德空间中的有界闭集都是紧的。

在直观上大致可以这样想象:距离空间中有无穷个点的集合,如果它可以被罩在越来越小的有限个开球里,这个集合的性质叫完全有界。在完全有界集合里的一个无穷序列,任给很小的开球半径,这覆盖着集合的开球数量也都有限,总有一个含有序列中无穷个点,它们在任意小的球里互相``靠近''着,这个无穷靠近的子序列是柯西列。这柯西列的极限如果是都在这空间里,这空间称为完备的;如果它都在那集合里,这集合则是闭的,这样性质的集合称为列紧的,也是紧的,意思是它很密实且被有限地覆盖。列紧性与紧致性在距离空间没有区别,在第二可数$ T_1 $空间里也是如此,在更一般的拓扑空间,它们并不等价。

紧致性说明无穷空间里的一些性质,可以通过开集表示的相邻性,只经过有限的集合来确定。在有限的世界里,我们用数学归纳法来推理,对参数1具有某种性质,在某一个自然数参数正确时,都能够推出对下一个数也拥有,则对参数为任何自然数的数学式也都拥有这个性质。

紧致性则在无穷空间中,划出一类在里面可以归纳推理的集合:开集说明其中的点都有某种的性质,如果有限个开集都具有某种性质,能够推出它们覆盖住的点也都有相同的性质,则能被开集覆盖的集合的点都有相同的性质。所以空间的紧致性是有限性之外最好的性质。

考一下你能用想象来指导证明的能力。

\begin{enumerate}
	\item 收敛的定义只是描写某一个点如何是一个无穷序列的极限。这定义并没有说明它是唯一的。实际上$ T_1 $空间(各点都有不包含对方的开邻域),因为区分能力差,一个无穷序列可能收敛到空间中任何一点。$T_2$空间(Hausdorff空间,任何两个点能有分属它们不相交的开邻域),则具有较强的区分能力。请证明:距离空间是$T_2$的。在$T_2$空间,一个无穷序列如果收敛,它的极限是唯一的。
	
	\item 用例4.3,4.4,4.6介绍空间和无穷序列($ t_n $)的性质,以及这些性质间的关系,请推出这序列在$ (C[a,b], d_1) $ 空间是完全有界的,在$ (C[a,b], d_\infty) $ 空间不是完全有界的。请描述这两函数空间里的开球是什么样的图像,为什么在$ d_1 $的距离下,可以用有限的开球能覆盖这序列,在$  d_\infty $ 的距离在却不能?请注意正确的想象是能够作为数学证明思路的图像。
\end{enumerate}

【扩展阅读】

\begin{enumerate}
	\item 维基百科,度量空间\url{Phttp://zh.wikipedia.org/wiki/\%E5\%BA\%A6\%E9\%87\%8F\%E7\%A9\%BA\%E9\%97\%B4}
	
	\item 程代展,系统与控制中的近代数学基础,北京:清华大学出版社,2007 \url{http://product.dangdang.com/9350967.html}
	
	\item Davis Edu, MetricSpace  \url{https://www.math.ucdavis.edu/~hunter/m125a/intro_analysis_ch7.pdf}
\end{enumerate}