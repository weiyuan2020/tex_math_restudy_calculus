\chapter{重修微积分10——算子}

算术是从给定条件和已知数,得出符合条件数值,计算的学问。算法用给定的条件,构造性地定义了从已知数到得数的映射。算术所在的数域仅仅是抽象空间包含的一个实例。近代分析把抽象空间作为给定条件,定义在空间的映射称为算子,研究它们的一般性质。

例如,迭代算法是用相同的子算法,把得数作为下次计算的已知数,一次次地迭代计算来逼近结果的计算方法。抽象空间里的压缩映像是能够应用于这类计算的算法。

距离空间 $ (X,d) $ 具有如下性质的映射称为\textbf{压缩映像} $ T: X \rightarrow X, \;$  $\; d(Tx, Ty)\leq ad(x,y) $,  $ a\in(0,1),\forall x,y \in X $。这里的$ d(x,y) $是x和y间的距离,如差值的绝对值及范数等等。应用压缩映像不断地迭代计算,能够逼近至多一个不动点。巴拿赫不动点原理说,压缩映像T在完备距离空间X中,有唯一的不动点$ x $,即 $  Tx = x $. 有了压缩映像T,在X中任取一点$ x_0 $,令$ x_{n+1} =Tx_n, n=0,1\cdots $,它将收敛于一个不动点$ x $.重要的原理都是简单和直观的。这个定理在分析中是个强有力的工具,以此可以证明空间和流形上的各种的反函数定理,它不仅可用来定性地证明常微分方程解的存在和唯一性,而且是解方程迭代算法的基础。

压缩映像是个算子,可以是线性也可以是非线性的。但在分析中,研究最多并最富有成果的是无穷维空间的线性算子。

为什么要研究无穷维空间的线性算子呢?因为物理的动态系统从初始状态开始的变化,微分方程由状态函数映射成微分关系和边界条件,工程系统由输入转变成输出,都可以看成一个算子作用在函数空间进行变换。出自叠加原理应用和近似的简化,这些算子也多是线性的。微分和积分作为算子是线性的,它们与其他因子组合而成的算子,例如傅立叶变换,拉普拉斯变化等积分变换,线性常微分方程,数理方程,这些数学工具都是无穷空间的线性算子。

数学上,代数关心的是集合中元素在运算映射下的性质。分析则研究这些代数运算在无穷空间中变动极限的性质。这就要考虑集合所在空间的拓扑性质,和了解在这些代数运算角度下空间的结构。上一篇,我们用有界线性泛函的内积形式,揭示了巴拿赫空间以及希尔伯特空间的对偶关系。这里要介绍线性算子的基本性质。

请注意,空间的线性,是集合中的元素对线性运算封闭,例如连续函数集合是线性空间,因为连续函数的数乘和相加仍然是连续函数。而算子的线性,则是算子的映射对空间上的线性运算保持不变的关系,即线性组合的映像等于组合中元素映像的线性组合,例如积分是在闭区间连续函数空间上的线性算子,线性常微分方程的系数可以是非线性的函数。线性算子必须作用在线性空间上,而线性空间上的算子可以是非线性的。

复习一下这里要用到的几个空间的概念。距离空间在集合任意两点中定义有距离,以此定义开球和邻域,生成空间的拓扑。赋范空间是线性空间,是以向量长度为范数导出了距离的距离空间。在这距离定义下,如果它对收敛还是完备的,则称为巴拿赫空间(Banach space)。希尔伯特空间是定义有内积的巴拿赫空间。这篇谈定义在赋范空间上的线性算子。线性代数课程讨论问题在欧几里德空间,它是有穷维的巴拿赫空间$ \mathbb{R}^n $,其线性算子在基底下表示为矩阵,它只是局限在有穷维的表现,我们对比地来介绍一般的线性算子。

上篇说过,对线性算子T,如果存在着一个正数c,对其定义域上所有的点都有$ \|Tx\| \leq c\|x\| $,则这个算子是有界的。这个c的下确界称为\textbf{线性算子}T\textbf{的范数}。

欧几里德空间上的线性算子都是连续的和有界的,这在巴拿赫空间未必成立。

\kaishu\setlength{\leftskip}{1em}

例10.1:闭区间[0,1]上连续函数集合C[0,1],以$ \|x\|_\infty = \max_{0\leq t \leq 1} |x(t)| $ 为范数构成巴拿赫空间。定义在C[0,1]到C[0,1]的微分算子D,是这空间上的线性算子,它的定义域是在 C[0,1]中所有连续可微函数构成的线性子空间。D不是有界的。因为函数序列{$ x_n(t)|x_n(t)=e^{−nt}, n\geq 1 $}在C[0,1]中,有$ \|xn\|_\infty =1 $,不难推出它是个有界的集合(这集合中任何两点的距离都小于2),但是$ Dx_n(t)=\frac{d}{dt}x_n(t)=−nx_n(t) $ 的集合,却是C[0,1]中的无界集。算子D将有界的集合映射成无界的像,所以它是无界的算子。

\songti\setlength{\leftskip}{0em}

线性算子是否有界,还取决于映射所在空间的拓扑。微分算子D定义在另一范数的空间,可以是有界的。

\kaishu\setlength{\leftskip}{1em}

例10.2:连续函数集合$ C[0,1] $以范数$ \|x\|_\infty $,构成巴拿赫空间。记函数x的导数为x’,连续可微函数集合$ C^1[0,1] $的范数定义为$\|x\|1= \|x\|_\infty +\|x′\|_infty$ ,它也是个巴拿赫空间。

微分算子D也是$ C^1[0,1] $空间到$ C[0,1] $上的线性算子,有
\[\|Dx\|_\infty =\|x′\|_\infty \leq \|x\|_\infty +\|x′\|_\infty =\|x\|_1,\forall x \in C^1[0,1]\]
这说明$ C^1[0,1]$空间中有界的集合,在算子D映射下在$ C[0,1] $上仍然是有界的,所以它是这空间里有界的算子。

\songti\setlength{\leftskip}{0em}

尽管赋范空间中线性算子不一定是有界的,但有界性和连续性却是等价的,甚至只要在某一点上连续,它们就有了全体的连续性和有界性。

在欧几里德空间,线性算子$ T $,用下面的内积式子,可以定义它的对偶算子$ T^* $,
\begin{equation}
	\left\langle Tx, y \right \rangle = \left \langle x, T^* y \right \rangle, \;\;\;  \forall x \in \mathbb{R}^n, \;\; \forall y \in \mathbb{R}^m
\end{equation}

用矩阵表示,$ T^* $是$ T $的共轭转置矩阵。对于赋范空间,我们也对线性算子用相同的方法定义其对偶算子。不过从空间X到Y的算子不一定对全空间都有定义,例如在微分算子T在连续函数空间$ C[0,1] $,只对其中连续可微的函数有定义。上述的定义只限在各自的定义域里。

算子$ T $的定义域记为$ D(T) $。如果$ D(T) $是稠集,即空间X中任何一个点,都可以表示为$ D(T) $中点序列的极限,$ T $则称为是\textbf{稠定}的。赋范空间X到巴拿赫空间Y上的有界线性算子,如果是稠定的,它可以唯一地(按连续性)延拓到整个空间,成为定义在X上的有界算子,且算子的范数与原来的相等。

如果$ T $的值域充满了Y的全空间,则称为是满的。如果$ D(T) $中的序列$ (x_n) $当$ x_n \rightarrow x,Tx_n \rightarrow y $,有$ x $也在$ D(T) $中,$ Tx=y $,则称$ T $是闭的。\textbf{闭算子}定义域中所有的点$ x $与对应像$ Tx $的组合$ (x, Tx) $,在$ X\times Y $空间中是个闭集。

如果$ T $是两个希尔伯特空间中的线性算子,T是稠定的则$ T^* $是闭的,如果$ T $还是闭的,则$ T^* $也是稠定的,而且有$ T^{**}=T $。

欧几里德空间上的线性算子,可以表达成矩阵A,其范数$ \|A\| $
是它与共轭转置矩阵相乘$ A’A $最大特征值的开平方值,它们的全体也是个欧几里德空间。相应的,赋范空间X到巴拿赫空间Y上的有界线性算子,所有这些有界线性算子$ L(X,Y) $,在算子的范数下也是巴拿赫空间。

微分方程可以看成算子T作用在线性距离空间的点上,等于另一空间的点(函数,初始或边界条件),例如Tu = v。应用算子理论研究微分方程,解的存在性对应着算子T有右逆,解的唯一性对应着算子T有左逆,所以T的逆算子存在意味着解的存在和唯一性。解的稳定性说,当参数、边界或初始条件变化很小时,解也应该变化很小,这对应着逆算子的连续性即有界性。在未知是否有逆时,稳定性的要求表达成算子的开映像性质,而微分方程解对初值一致连续性则表达成算子族的一致有界性。

这些问题,对于巴拿赫空间X到Y的线性算子T,都已经有了很好的答案。

\textbf{开映像定理}:如果T是闭的和满的,对于任意小的 $ \epsilon >0 $,有相应的 $ \delta >0 $,使得
\begin{equation}
	\forall y\in Y, \; \left \| y \right \| < \delta \Rightarrow \exists x \in D(T), \; \left\|x\right\|<\epsilon , \; y=Tx
\end{equation}

简言之,开集的像是开的。

\textbf{巴拿赫逆算子定理}:如果T是闭的,满的,一一对应的,则它的逆算子存在且是有界的。

\textbf{闭图像定理}:如果T是闭的,定义域是X全空间,则T是有界的。

\textbf{共鸣定理}:如果一族有界线性算子在X上是逐点有界的,那它们也是一致有界的。

上述这几个是泛函分析中线性算子的基本定理。它们都还有在更广泛的线性距离空间,附加上一些条件的版本。有兴趣请看泛函分析的教科书。

线性算子的这些性质,让线性微分方程成为描述世界强有力的工具。在这种线性描述下,逆算子的存在,证明了一切的变化都可以由已知的原理、参数、边界和初始条件唯一地确定;逆算子的连续性保证了一切的误差都是可以无限地消减。这个关于无穷过程线性数学利器的成功应用,让人们相信世界是确定性的,无穷可分的,差不多是线性的,几乎忘记了为了能够应用这个利器,曾经省却了一些细节,作过了一些假设,即使非线性的研究也只往这方向靠,忽略本质不同难以想象的部分,直至非线性动力系统以混沌、分叉、孤立子突兀在眼前,打破了幻想,在数学上揭示了系统上不确定的机制。

三百多年前,微积分以函数为阶梯从无穷小分析的思路,把人们带进了想象中的无穷世界。由线性联系着微观机制和宏观的表现,线性系统的叠加原理所惠,让它成为研究动态和连续系统最强有力的工具。短短的三百年时间,研究函数科学的分析,成为数学最大的分支。在这无穷可分几乎是线性的世界里,数学分析是撰写自然律法的笔墨文书,是从事理工研究必不可少的工具。我们对世界的认知,其实是符号的象征和想象的产物,数学工具极大地影响着研究者对事物构造的想象和规律的理解,进而推及大众。计算机的出现,将可能改变世界的图像,影响着数学研究方向和对世界的认知。世界也许将回到有穷分立的结构,非线性将是主流,复杂和不确定系统或成为富饶的主题。但无论怎么改变,数学都是人们用逻辑来及远的工具,指使计算机的方向,描绘世界的画笔。现在的计算机在科研中,还基本是分析计算的工具,图像表达的机器和记忆搜索的助手。人们还未找到代替叠加原理超越线性,组合分立研究结果的方法,还在等待着一个革命性的思想,来指引怎样用计算机来描述,理解和控制我们的世界。在这之前,数学分析仍然统治着物理和工程的世界,理科生还离不开这时代战士必备的这个武器。



【扩展阅读】

\begin{enumerate}
	\item 关肇直等,张恭庆,冯德兴,线性泛函分析入门,上海科学技术出版社,1979

	\item 程代展,系统与控制中的近代数学基础,北京:清华大学出版社,2007 
	\url{http://product.dangdang.com/9350967.html}


\end{enumerate}

转载本文请联系原作者获取授权,同时请注明本文来自应行仁科学网博客。
链接地址:http://blog.sciencenet.cn/blog-826653-893883.html 