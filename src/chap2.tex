\chapter{重修微积分2——收敛}
无穷序列可以用来表示一种趋向。其思想仍然与归纳法一样,企图用已知来推测未知。这里是用有穷的序列项来推测无穷之处的结果。只不过数学归纳法,只能在有穷的世界里漫行,这里需要一个假设,才能用逻辑跨过边界。

大致地说,无穷序列作为数学的模型,在这无穷过程中,当后来的项越来越相像,如果这无穷过程指向一个实无穷的极限,设定的含义是:这极限与有穷过程里的项,也将会是越来越相像,以致难以区分。在这种情况下,我们有把握确定这个极限的性质,可以用这个无穷过程来定义或确定这个极限。

这里有两个问题。一是这序列能否指向一个数学实体,二是什么叫做“相像”。

第一个问题,不外乎三种情况。一是不能。那就飘过。但如果是都不可能,那就不用谈极限了,微积分是个梦,大家仍旧玩算术。二是指向未知的实体。我们也许可以用模型里“相像”这性质来定义它,叫做扩充。这押后再谈。三是指向这空间里已有的实体,称之为收敛的极限。因为它与序列中的项同在一个空间,它们在空间里的关系可以用来描述“相像”的含义。

以上所说并不限于实数,适合于包括函数、事件,以及集合元素的无穷序列。不同的“相像”含义,确定了不同收敛和极限的含义,这将在以后的篇章里展开。请记住这里的图像,作为分析中直观想象的基础。这一篇,我们先谈最简单的情况——数列。

对于数而言,现实的计算只能区分有限精度的数。按照某种精度画个圈,一个无穷的数列,除了前面有限多个外,如果此后所有的项终将全部落入这个圈里,按照这数学模型的假设,它指向的极限,也是在这圈子里。在这圈里任何两个数,按这精度看不出区别,在圈里取任意一个数当作那个极限也是如此。如果总能以任何精度做到这一点,也就能以任何精度确定这个终极的数。这个无穷数列就对应着这个数。

如果这个极限在讨论的空间上存在,这数列称为“收敛”到这个极限。

用数学语言表达无穷数列$ (x_n) $收敛于a是: $ \lim_{n\rightarrow \infty} a_n = a $ 或者写成 $ a_n \rightarrow a , \mbox{ } (n \rightarrow \infty) $ ,在不混淆的情况可以略去括号里n的走向。上述收敛的定义用 $ \varepsilon -N $ 语言和数理逻辑符号表达是:
\begin{equation}
	\forall \varepsilon > 0, \exists N \in \mathbb{N},\forall n ( n>N \Rightarrow |a_n-a| < \varepsilon )
\end{equation}

这定义只是说,如果有个数符合这描述,则称数列收敛于它。要认为这个无穷数列确定了这个数,需要证明它是唯一的。作为数学模型的要求,我们还必须证明,收敛的极限与空间里的数做四则运算,其结果与用数列中的项参与同样运算的数列极限是一样的。这些证明都不难,最后都归结为:要验证的差异随着N足够大也会小于任何正数。即这个差异的无穷数列无限地趋向0。所以在极限时等于0. 停!在这句结论之前,全部是对有穷数列的推算,这无穷数列要走过所有的项,才能得出相等的结论,这需要一个假设,才能越过这个逻辑的间隙:无穷必须是可以完成的!这样无限减小又始终存在的差异才会消失,让无穷序列的桥梁能够搭上有限世界的对岸。这是被许多人忽略,以致缺乏动力去改变有穷世界里的直观。

从收敛的定义不难看出,如果无穷过程收敛,那么在这过程中走得足够远的项之间的差别也会足够小,即
\begin{equation}
	\forall \varepsilon > 0, \exists N \in \mathbb{N},\forall n \forall m ( m,n>N \Rightarrow |a_n-a_m| < \varepsilon )
\end{equation}

这样的数列 $ (x_n) $ ,叫做“柯西列(Cauchy sequence)”。能够收敛的数列都是柯西列。

回头来看格蓝迪级数1-1+1-1+1-1+…,这个无穷过程无法确定任何数值,其有限和构成的无穷数列不是柯西列,不能收敛。欧拉和在收敛意义上不成立。

并非所有柯西列都能收敛,它取决于所在空间的性质。例如,$ \sqrt{2} $

取越来越多位数的数列1.4,1.41,1.414,… 是柯西列,但它在有理数空间不收敛。即这个数列极限在有理数中不存在。

能够让所有柯西列都收敛的数学空间,叫做“完备的(Complete)”。微积分发明之后,数学家又发现了大量未知的实数和并严格修补了实数的定义,从而证明了实数是完备的。这表示只要数列是趋于相互靠拢的,就一定有极限,这个意义很重大。初等微积分是建立在实数完备性的基础上。

实数在近代,是用有理数上戴迪金分割来定义,把有理数集分割成上下两个集合,让上集合中所有的有理数,都大于下集合的;无理数被定义为,填充这种分割中的“间隙”。可以证明,这样定义的无理数和有理数,构成的实数,在收敛的意义下是完备的。

实数在传统上,看成整数加上一个无穷的小数。一个无穷小数,取越来越多位数的过程,是个柯西列,如果定义这个无穷过程表示的数学实体叫做“实数”,这时,实数可以看成是有理数的完备化扩张。

不论是怎样构造的,它们是等价的,实数是完备的,所有的柯西数列在实数上都收敛。

下面是几个微积分上熟知的定理,说的都是实数的性质,可以证明它们是互相等价的:

\kaishu
\setlength{\leftskip}{1em}
\begin{enumerate}
	\item 实数是完备的。
	
	\item 实数任意上(下)有界的集合,一定有上(下)确界。
	
	\item 单调有界的数列有极限。
	
	\item (区间套定理)闭区间序列$ ([a_n, b_n]) $	,如果$ [a_n, b_n]\supset [a_{n+1}, b_{n+1}], \forall n \ge 1  $并且 $ (b_n-a_n)\rightarrow 0 $ ,则必定存在着唯一的实数c有 $ c \in [a_n, b_n],\forall n \ge 1 $ ,并且 $ a_n \rightarrow c $ 和 $ b_n \rightarrow c $ 。这个闭区间序列叫作“区间套”。
	
	\item (有限覆盖定理)有界闭区间I上的任何开区间覆盖族中,必定有一组有限的开区间覆盖I。
	
	\item (列紧性定理)任何有界数列必定有一个收敛的子列。
\end{enumerate}

\songti
\setlength{\leftskip}{0em}
采用戴迪金类似的方法,是否还能发现实数间的空隙?从实数的完备性,可以证明了这样的空隙不存在。这也称为实数是“连续的”。

任何两个有理数,无论之差是多小,其间都有不可数个实数。只有夹在一个无穷地不断增大的有理数列,与另一无穷地不断减小的有理数列中,当它们之间的差值趋向0时,其“缝隙”才只够容下一个实数。

实数的完备性是微积分的基石。与此等价的上述性质,在微积分中扮演了重要的角色。

康托尔用对角线法证明了,实数是不可数的。从实数完备性上,也不难证明实数是不可数的。

\kaishu
\setlength{\leftskip}{1em}
用反证法。假设闭区间[0, 3]里的实数可数,那么可将它们标记为:$ x_1, x_2, x_3, \cdots  $。将$ [0, 3 $]等分成三个区间 $ [0, 1], [1, 2], [2, 3], x_1 $必然不在其中之一,记这区间为$ I1 = [a_1, b_1] $,再将$ I_1 $等分成三个区间,同样必有一个子区间不含$ x_2 $,记这区间为$ I_2 = [a_2, b_2] $,如此进行下去,得到一个区间套$ I_1, I_2, I_3, \cdots $。根据区间套定理,存在着一个实数y在所有这些区间中。但从区间的选取中知道,$ x_n $不在$ I_n $中,所以实数y不是$ x_1, x_2, x_3, \cdots $中任何一个。这就和假设相矛盾。所以实数是不可数的。

\songti
\setlength{\leftskip}{0em}
实数必须是不可数的,才不会与它的完备性相冲突。在可数的世界里,没有微积分。

对于古希腊毕达哥斯派的疑问:无穷循环小数0.999…的每一个截断都小于1,它怎么最终会等于1?从对于每一个截断所有n都成立的命题,得出对全体也成立的结论,是有穷世界直观的错觉。持潜无穷的观点,0.999…只是个无穷的过程,这小于的关系永远存在,当然不会等于1。但是这里的“所有”,指的是对每一个有限的数n,而不是包括了它们无穷全体的“最终”。就像上篇的数学归纳法的例子,这个对所有n的推理,只在有穷世界里是对的。不能应用于无穷。这是必需改变的观念。按实无穷的观点,这个无穷的过程是可以达到它的极限,也就是等于1,这小于的关系,因达到极限而终结为相等。收敛的极限可以等于比数列中每一项都大(小)的上(下)确界。

柯西和魏尔斯特拉斯之后的微积分,为了避免过去对实无穷的简单解读,在能够用潜无穷观念解释的地方,尽量避免使用实无穷。但对收敛的极限,逻辑上毕竟不能绕开实无穷的观念。

也许大家觉得上面所说的内容很浅,早在学习微积分时就知道了。好,考个对收敛理解的问题。

\kaishu
\setlength{\leftskip}{1em}

张三找李四寻仇,拿个炸弹设了一分钟后扔向李四,李四半分钟后扔回,张三在四分之一分钟后又扔向李四,如此往复,问最后炸弹在谁的手里爆炸?

(停下思考5分钟,再往下看,再思考。)

如果知道某个时刻落在这个无穷往复过程的哪一个时段,就知道炸弹在谁的手里。虽然这是个无穷的过程,但爆炸的时刻终将会到来。这些时段的累加为1/2+1/4+1/8+……,级数收敛到极限1。但时刻1在这些描述炸弹位置的时段里吗?这个数学模型能描述现实的情况吗?

\songti
\setlength{\leftskip}{0em}

【扩展阅读】

\begin{enumerate}
	\item 实数理论 \url{http://210.26.16.17/ziyuan/38/analysis/precis/Acrobat/21.pdf}

	\item 实数的完备性 \url{http://210.45.128.5/jpkc/shuxiefx/sxfxky/hky.pdf}
\end{enumerate}
