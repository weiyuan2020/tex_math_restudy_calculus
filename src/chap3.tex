\section{重修微积分3——拓扑}

上一篇探讨了实数收敛的概念,用无穷级数来确定一个数。很容易把它扩展到无穷的函数序列求极限的问题。在初等微积分里,这是对函数变量的每个值逐点来考察,对每个固定的变量值,这无穷序列对应着一个函数值的数列,如果所有点对应的数列都收敛,那就认为这无穷函数序列收敛,它的极限函数在每个点的函数值是相应数列的极限值。这样函数序列的极限称为逐点收敛的。这个方法被牛顿引入后,广泛地应用,它虽然可行,但在微积分进一步研究时又遇到种种麻烦,于是又附加了许多条件,如“一致连续”,“绝对可积”等等,最后弄得微积分繁杂不堪。能不能把整个函数看成一个数学空间里的一个点,把这些条件都看成空间里的性质,从一个统一的角度来研究收敛极限的问题?这便是这一篇要介绍的概念。

现代的数学建立在比实数更加抽象的集合论基础上,应用于更广泛的空间。要将定义在实数上一元函数微积分的本质说清楚,推广到多元函数,函数逼近,泛函,随机过程,乃至各种抽象数学结构的集合上,我们要了解集合元素间联系的结构,这样才可能描述变动个体的走向,空间的性质,进而谈及趋近、收敛和极限。

微积分是基于无穷逼近极限的数学。收敛描述的是变动差别越来越小,直至微不可察的数列表现。收敛极限的存在,取决于实数的完备性。所以无穷逼近过程的含义和结果,依赖于它所在数学空间的性质。传统微积分是建立在实数空间R

的性质上。这些实数性质是人们在习以为常的计算经验中总结“发现”出来,又用一些更简单基本的原理来证明的。

在现代数学之前,数学一直被认为是研究真理的学问,直到百多年前,哲人们还都认为不但如此,而且是绝对的真理。现在看来,这些真理也许不过是:习惯成了自然的经验在逻辑上自洽的表现。几千年实践认知形成有穷数学的直观与那时的理论吻合无间。无穷则无法验证,除了凌空而来的假定和逻辑自洽之外无所凭借,一切概念和引为依据的原理都需要精确的定义引入,一切的争执最终也只能根据这些定义和假设,用逻辑作出判决。直觉、经验、试验和公认之事,对数学都不足为凭。数学的价值在于有用,在虚构的空间里,以严谨的逻辑铸造出一把犀利的思想宝刀,当你用对了,无数的实践难题可以迎刃而解。

要想自如地在在抽象的无穷空间中行走,你需要正确的直观想象。前面两篇只是个导引,让你意识到必须纠正过去的直觉,才能放开心怀接纳无穷空间里的真实。在“无穷”篇里数学归纳法的例子,证明了在有限情况都成立的论断,在无穷时未必成立。要走出有限的误区,才能走出局限。在“收敛”篇里0.999…=1的分析,说明了极限的含义。这是微积分数学模型的最基本假设,无穷过程达到彼岸的立足点。所有收敛的极限都是依相同的原理,将差异的不等式在极限处变成了等式。除非你曾经震惊思考改变过,如果初读前面这两个例子,没有经历过困惑,思考到终于理解了,你大约还没有走出有穷世界的藩篱。建议你回去走通了再来,否则学过的微积分知识只是别人说的,不是自己会的,此后三篇让你构筑无穷空间想象的介绍,就可能格格难下不能吸收,无缘看见彼岸。

直觉是模糊的联想,并非精确可靠。别人说的风景和你所想的也许不是同一个地方。只有踏在共同之处才是真实的一样。想象必须依附于证实过的判断作为立足点,只有在坚实的桩脚附近延伸,才靠近真实,过度延伸的浮想只是梦呓。可靠的直觉来自大量被证实过的事例,它们编织成了图像,越多的事例,才有越清晰的直觉。在现实有穷的世界,这些事例来自日积月累的学习和实践,判决正误唯有实验。在抽象无穷的空间,可靠的事例来自符合定义的例子和严格证明过的判断,在这里逻辑是唯一的判据。你必须记忆概念的定义,消化例子,纠正限于有穷世界的直觉,才会形成在无穷空间里的正确直观。

这一篇黑体字强调的概念,如开集、闭集、邻域、收敛、稠集、连续等等,你可能已经认识,不过在初等微积分学的是在实数上的定义,这儿是从抽象集合的高处俯视。这篇把点集拓扑的基本概念串起来介绍,有足够的信息让你可以依想象,走通从定义到例子的逻辑。希望你从过去的认知起步,以这里的定义来消除局限,用例子来校正偏差,在走通的过程中重建图像。花费时间逐个概念走过,你将会开始有看清无穷空间景象的基础。

好吧,抽象!讨论问题的空间一切属性都需要定义。忘掉实数、无穷、收敛、极限种种课堂灌给我们前人发现的真理,从简单干净的抽象集合开始。上一篇说,无穷序列作为数学模型来描述收敛的观念,序列中的项以及极限之间的“相像”性质,决定了收敛和极限的含义。好的。对集合中的两个元素,比如说是两个人,你能从其一了解另一吗?不能。因为元素只是集合中抽象的个体,没有它们间关系的信息。如果说知道它们同在某一个子集里,好比说知道两人同是一族,你就能知道,他们比不在这子集里的外族人有某些更相像。在这里用子集表达其中的元素具有某些共同的属性,这样的子集称之为开集,它们的并和交满足某些封闭性。集合上有了这样的一组开集,便能以此判断元素间属性是否相近。可以用来判别相邻性的这组开集,称为集合上的拓扑,它们组成了讨论无穷逼近问题的数学空间。

\kaishu
\setlength{\leftskip}{1em}
考虑集合X上的一个子集族$ \tau $,其元素是X的子集,如果它们任意多的并,及有限个的交都属于$ \tau $,空集和X也属于$ \tau $,那么 $ \tau $ 中元素称为\textbf{开集}, $ \tau $ 称为X上的\textbf{拓扑(Topology)},组合$ (X, \tau)$称为\textbf{拓扑空间},在简略$\tau$不致混淆时,也可直接称X为拓扑空间。

\songti
\setlength{\leftskip}{0em}

在任何拓扑,空集和X都是开集,只具有这两个开集的拓扑,称为\textbf{平凡拓扑}。取X上的一组子集,包括空集和X本身,定义它们为开集,进而将它们任意多的并,及有限的交,都视为这空间的开集,则生成了一个拓扑$\tau$。给几个实例来帮助想象。

\kaishu
\setlength{\leftskip}{1em}

\noindent
例3.1:定义实数上的开区间为开集,在它们间进行任意多的并和有限个交运算,依定义这生成了实数上的拓扑。通常实数空间指的就是以此为拓扑的空间。

\noindent
例3.2:在闭区间$ [a,b] $上所有连续函数的集合$ C[a,b] $,对任意的 $ f\in C[a,b],\;\;\delta \textgreater 0 $ 
,定义集合 $ \{ g \in C[a,b]\; |\; \max_{a\leq t \leq b} |g(t)-f(t)|<\delta\} $ 为开集,它们任意多的并和有限个交,生成了$ C[a,b] $空间上的一个拓扑。

\noindent
例3.3:集合$ X=\{a,b\} $,定义空集$\phi$,单点集$ \{a\} $和X为开集,它们构成X上的一个拓扑。

\noindent
例3.4:将X上所有子集都定义为开集的拓扑,称为\textbf{离散拓扑}。它也可以由定义任何单点集都为开集来生成的。

\songti
\setlength{\leftskip}{0em}

拓扑空间$ (X,\tau) $中集合X称为空间,它的子集简称为集合,X的元素称为点,点之间的相邻关系,由$\tau$来确定。两个点,如果同在拓扑空间的一个开集里,认为它们是相邻,分属两个不相交的开集,认为是分离;包含了点x的开集,称为x的\textbf{开邻域},包住x开邻域的集合称为x的\textbf{邻域}(neighborhood),寓意是与这点某些属性相近的点都在这里。

那么,怎么用拓扑来描述趋近状态呢?

对于拓扑空间X的一个点x和序列$ (x_n) $,如果对于x的每个邻域U,都有对应着它的一个正数N,当$ n \ge N  $ 时,都有$ x_n \in  U $,则称$ (x_n) $\textbf{收敛}于x。考虑到点的邻域含义,这意味着,对这空间中拓扑所考虑的属性上,这序列后面的点终将与点x是密不可分的。

对于拓扑空间X中一个点x和集合A,如果x的每个邻域都包含有除了x外,还有A中其他的点,则x称为A的一个\textbf{聚点}(cluster point)。例如:在实数空间中,0和1是开区间 $ (0, 1) $的聚点;任何一个无理数都是有理数集的聚点。

显然,序列收敛到非序列中的点x,是序列所在集合的聚点;但集合A的聚点,未必有集合中的序列收敛于它。如果集合中的点都有一组可数的邻域,这点所有邻域都包含这组的某些成员,这样的拓扑空间称为\textbf{第一可数的}(first countable)。第一可数空间集合中的聚点都有集合中的序列收敛于它。在分析应用中的聚点都是用序列来逼近的,所以应用中的拓扑空间都是第一可数的,它也称为可数性的第一公理。

开集的补集称为\textbf{闭集};包住集合A的所有闭集的交,是包住A的最小闭集,叫做A的\textbf{闭包}。闭包也是包括了A和它所有聚点的集合。实数是有理数的闭包。闭区间[a.b]是开区间(a,b)的闭包,闭区间[1, 3]是集合 $[1, 2)\cup $  $(2, 3) $的闭包。

对拓扑空间X的子集A,如果任何开集都包含有A中的点,则称A是X中的\textbf{稠集},意味着X中的点的任何邻域,都含有A中的点,即可以用A中的点来逼近。如果X有个稠集是可数的,则称这空间是\textbf{可分的},X中的点都有这稠集中的一个序列收敛于它。有理数是可数的,它是实数上的稠集,所以实数空间是可分的,每个实数都有一个有理数的数列收敛于它。所有实数都是有理数集合的聚点,实数空间的拓扑是第一可数的。

同一个集合X上可以定义不同的拓扑。平凡拓扑是最粗的拓扑,离散拓扑是最细的。开集设定了一个点的邻域,它意味着里面其它点与这点相类。邻域之外就有某些的“不相类”,所以空间的拓扑区分的能力也很重要。如果对两个点,总有一个点开邻域不含另一,称为$ T_0 $空间;它们各自都有不包含对方的开邻域,是T1空间;如果两个点能有分属不相交的开邻域,叫$ \bm{T_2} $\textbf{空间},这也称Hausdorff 空间 ;$ T_3, T_4 $则是说明点与闭集的区分能力。后者包括前者。过粗的拓扑,让空间中的点含糊难分。过细的拓扑也孤立难同,则难以找到点邻域里的其它点来推测它的性质。应用中大多数具有丰富性质的拓扑,既不会太粗也不会太细。现代分析是建立在点集拓扑研究的基础上。这样可以从简单拓扑结构开始,层层加细,从不断丰富拓扑空间性质的研究中,探究各种特性的本质。

对于微小的变动过程,如果集合上的函数,对应的值的变化也是微小的,即$ x $趋向$ a $时,$ f(x) $趋向$ f(a) $ ,这是一种叫做“连续”的对应关系。用拓扑的语言来描述函数$ f(\cdot) $在点$ a $处连续,它的定义是函数值$ f(a) $的每个邻域,都包含了$ a $一个邻域的映像(即这邻域所有函数值的集合)。说函数$ f(\cdot) $ 是连续的,指它对定义域中所有的点都是连续的。

从拓扑空间X到Y的连续函数意味着,Y的每个开集U,它的原像 $ f^{-1}(U) $ 在X中也是开集;闭集V的原像$ f^{-1}(V) $也是闭集。连续函数把它对应的两个空间的拓扑结构连系起来了。

两个拓扑空间之间如果有个一一满映射$ F $,$ F $和$ F^{-1} $都是连续的,则称$ F $是同胚映射,这两个拓扑空间称为是\textbf{同胚}的(Homeomorphism)。想象拓扑空间X就像一个有弹性的几何体,同胚映射$ F $就好比将这几何体拉伸压缩变形,形成另外一个几何体拓扑空间$ Y $。连续函数$ F $在这变形中,让所有点的相邻关系都没变。如果$ F $的映射不是满的,则说$ F $将$ X $\textbf{嵌入}(embedding)到$ Y $空间中。传统的拓扑学,用同胚映射和同伦映射对空间进行几何分类和代数特征的研究。分析应用上,将研究的拓扑空间同胚映射或嵌入到熟悉的空间中,可以比较直观地了解它的性质。

上面所说的是在集合的基础上,抽象地定义具有收敛概念的数学空间。这里尽量用简洁清晰的语言表达概念的定义,如果仍有未解之处,建议查看点集拓扑的教科书来加深理解。你的努力,将给了你一双能够看到无穷空间的眼睛。

现在检验你对抽象定义的理解力,请思考下面的概念题。

对拓扑空间X中的集合M,X的所有开集与M的交集定义为M中的开集,构成M的拓扑,称M为X的\textbf{拓扑子空间}。M是个拓扑空间,它继承了X的拓扑。例如实数R是3维欧几里德空间$ \mathbb{R}^3 $的拓扑子空间。

开区间$ (0, 1) $继承实数空间的拓扑,问在这拓扑子空间里$ (0, 1) $是闭集吗?它还是开集吗?在这里的有界数列都收敛吗?这子空间是完备的吗?这拓扑子空间与实数空间是同胚的吗?

【扩展阅读】

\begin{enumerate}
	\item 维基百科,拓扑空间\url{http://zh.wikipedia.org/wiki/\%E6\%8B\%93\%E6\%89\%91\%E7\%A9\%BA\%E9\%97\%B4}
	
	\item Stephen Willard,General Topology, Addison-Wesley(1970)
	
	\item Renzo’s Math 490,Introduction to Topology  \url{http://www.math.colostate.edu/~renzo/teaching/Topology10/Notes.pdf}
\end{enumerate}

