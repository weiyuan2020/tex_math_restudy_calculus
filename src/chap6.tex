\chapter{重修微积分6——微分}

\kaishu\setlength{\leftskip}{1em}

芝诺“飞矢不动”的悖论说:飞行的箭,每个时刻都占据了一个确定的位置,这意味着它不会同时存在其他的位置,箭矢的位置固定,所以它在这时刻是静止的。依此推理,飞行的箭在任何时刻都是静止的,所以运动在逻辑上是不可能的。

\songti\setlength{\leftskip}{0em}

对于这个悖论,有不同的解答。黑格尔认为运动就是一对矛盾,每个时刻飞矢是既在这个位置又不在这个位置上,用辩证法回避了形而上学的挖掘。康德认为时间和空间并非事物的属性,而是我们感知事物方式的属性,这个矛盾是我们过去时空观念的疵瑕。休谟否认时空的无限可分性,以此也可以给出有穷时空的离散化解释。而牛顿坚持了时空无穷可分的观点,用微积分给予近代的解释。从而也让时空无穷可分的假设变成了公认的真理。

运动在直观上是个时间段上位移的现象,当一个物体在时刻$ t_0 $到$ t_1 $的时段,从位置$ x_0 $到了$ x_1 $,如果$ \Delta t = t_1-t_0\neq 0 $时$ \Delta x = x_1-x_0\neq 0 $,我们说它是在运动。物体在这时段的速度为$ \frac{\Delta x}{\Delta t} $,意思是位移对时段里时间流逝的变化率。物体时刻$ t_1 $在位置$ x_1 $,这个信息,不足以判定它是静止还是运动的。只要$ \Delta t > 0 $,速度 $  \frac{\Delta x}{\Delta t} \neq 0 $ ,在 $ [t_0,t_1) $时段都是在运动,牛顿把这确信是运动的区间无限缩小,当 $ \Delta t \leftarrow 0, \frac{\Delta x}{\Delta t} \leftarrow v $ 时,用这个无穷过程的极限,把运动和速度的概念扩展到时刻和位置点上,定义时刻$ t_1 $,物体在位置$ x_1 $处的速度为$ v $,只要$ t_1 $时刻(或$ x_1 $位置上)$ v\neq 0 $,说明这时在这一点也是在运动。

这区间里的运动和速度,从单纯逻辑上不能推出在$ t_1 $和$ x_1 $也是如此,这里还需要一个假设。他的信念是,运动的概念在这无穷过程推到了极限,都应该保持不变,在数学上的表达是:运动速度是连续,也就是说它不会在同一时刻有不同的值。牛顿用速度加速度的极限定义和力学第二定律,规定速度是连续的,而力可以是不连续的。在牛顿的力学世界里,运动是一阶可导的位置函数。

当计算区间无限缩小到达极限时,时段变成了时刻,位移变成了位置,这个变化实质是,无穷的过程用它的极限值来代表了。比值的无穷过程说法(这是柯西略微修正了牛顿无穷过程的比值说法),赋予速度新的含义,称之为在这一点的导数。微积分里用了莱布尼茨在数学上不严谨但应用上很直观符号,记为:$ dx/dt $.

这个无穷解释的观念影响至今,有了微积分这个利器,从此人们慢慢习惯世界是连续、无穷可分、确定性的,甚至是线性的了。这再次体现了康德的名言:“理性为自然立法”。

实数域上函数$ f(x) $ 在某一点的导数$ f′(x) $ 是函数值在这一点的变化率,它的直观几何图像是函数在这一点切线的斜率。函数的微分与变量的微分是一种线性关系$ df(x)=f′(x)dx $,这让人们构造数学模型时可以应用叠加原理,在近似时用差分来计算,因此被广泛地应用。

实数或复数值函数在所有点的导数,构成了导函数。所以对函数求导,可以看成函数空间的一种线性变换。记微分算子$ D=\frac{d}{dx},Df(x)=f′(x) $,它是作用在巴拿赫空间的一个线性算子。让我们从线性代数的角度来看,这线性算子将怎么分解所在的空间。

在线性代数中,我们知道线性算子在线性空间中有特征值,特征向量的子集,张成对这算子的不变子空间,全体构成线性空间的基,所有的向量都可表示为这些特征向量的线性组合。微分算子特征向量和它们张成的子空间对导数和积分运算封闭,微分方程在这里表现成向量之间的代数关系式,我们可以用它来解微分方程和逼近。下面考察怎样应用特征向量的例子。

显然,指数函数$ e^{−iat} $是微分算子$ D $的一个特征向量,这里$ i $是虚数符号,$ a $是任意复数,$ t $是实数变量。对这特征向量,$ D $的特征值是$ -ia $。取任意一组这样的特征向量,它们的线性组合是微分算子$ D $的不变子空间。但是仅仅如此的应用不多,我们更关心的是能否有个可数的正交基张成希尔伯特空间,让它里面的函数都能表示成无穷级数的和。这就和空间的拓扑性质有关了。

考虑希尔伯特空间$ L^2[0,2π] $,那么$ D $的特征向量集合$ {e_k=\frac{1}{\sqrt{2\pi}}e^{−ikt}|k\in \mathbb{Z}} $便是这空间上的一个正交归一基。

让我们首先来验证正交归一性。对于$ L^2[0, 2π] $空间,它的内积定义是$ \langle f(\cdot),g(\cdot)\rangle = \int_{0}^{2\pi}f(t)\overline{g(t)}dt $ ,对于任意整数$ m,n $ ,我们有:
\begin{equation}
	\left \langle \frac{1}{\sqrt{2\pi}}e^{-imt},\frac{1}{\sqrt{2\pi}}e^{-int}\right \rangle =\frac{1}{2\pi}\int_0^{2\pi}e^{-imt}e^{int}dt =\frac{1}{2\pi}\int_0^{2\pi}e^{-i(m-n)t}dt = \delta_{mn}
\end{equation}

这就证明了它们是正交归一的。空间中向量$ f(\cdot)\in L^2[0,2\pi] $ 在$ e_k $上的投影是:
\begin{equation}
	\left \langle f(\cdot),e_k \right \rangle = \frac{1}{\sqrt{2\pi}}\int_0^{2\pi}f(t)e^{ikt}dt
\end{equation}

这是大家熟悉的函数$  f(\cdot) $傅立叶系数的复数形式(若将复数展开成余弦和正弦正交基,则系数乘一个常数因子)。函数$  f(\cdot) $对这组向量的分解是傅立叶级数,不难证明这个傅立叶级数收敛于$  f(\cdot) $。所以它们构成了$ L^2[0, 2π] $空间上的基。经典的傅立叶级数,就是建立在微分算子$ D $一组在$ L^2[-T,T] $空间正交归一的特征向量上。这组可数的基张成了$ L^2[-T,T] $希尔伯特空间。

注意到微分算子$ D $,有不可数的特征向量$ e^{−iat} $ ,所以它们在无穷序列表达下可能是线性相关的。这取决于它们所在的空间。

是不是所有希尔伯特空间中的点都能表达成无穷级数?也就是说,是不是它们都有可数的基?答案是否定的。

\kaishu\setlength{\leftskip}{1em}

例如:对于函数定义内积为 $ \left \langle f(\cdot),g(\cdot) \right \rangle = \lim_{T\rightarrow \infty}\frac{1}{T}\int_{-T\pi}^{T\pi} f(t)\overline{g(t)}dt $,它构造了一个希尔伯特空间 $ L^2(-\infty, \infty)* $,对所有的实数$ s, t $的函数 $ e_s(t) = \frac{1}{\sqrt{2\pi}}e^{-ist} $都是这空间上线性算子$ D $的特征向量,不难验证它们是正交归一的,这组向量是不可数的。

\songti\setlength{\leftskip}{0em}

$ L^2[-T,T] $是可分的希尔伯特空间,里面的函数可以用傅立叶级数来表达(在$ L^2 $积分意义下收敛,级数展开几乎处处逐点收敛于它)。而$ L^2(-\infty, \infty)* $ 这希尔伯特空间是不可分的,所以这里的函数不能用傅立叶级数来表达。例子里那组向量是个不可数的正交归一基,这空间里的函数可以用积分变换来表达对这组基的分解和线性组合。从内积公式得到傅立叶变换,即是对这组基分解的分布函数;对基向量分布性分解的线性组合可直接写出傅立叶变换的反演。这提供了一个通俗的直观解读。更深入的探讨,诸如无穷区域的积分,无穷小分解系数分布函数的表达,积分的线性组合表示,及扩充到广义函数等等数学细节,在Sobolev空间可以得到更严谨的解读。

数学是直观想象在逻辑上精确化的学问。希尔伯特空间的研究,源自狄拉克对量子力学算符的表达。狄拉克非常注重数学上形式的美,简洁的美,他以此扩充了许多直观概念的应用场合,取得十分漂亮的结果。但在无穷世界的想象,还是需要用精确的逻辑来校正。1927年冯·诺依曼、希尔伯待和诺戴姆的论文《量子力学基础》,纠正了狄拉克缺乏严谨的不足。

在早期的泛函分析研究,特别是在物理应用中,希尔伯特空间指的是可分的完备的内积空间,即这空间有可数的稠集。上面的例子说明并非都是如此的。\\
\\
\\

大家已经熟悉在$ \mathbb{R}^n $ 空间上的微分,怎么将它推广到往整体看不是那么“平整”的空间?先看看平面几何是怎么使用的。我们生活的大地实际上是地球球面上的一部分,把这个局部当作2维的欧几里德空间,或者说映射到$ \mathbb{R}^2 $ 空间。每一个局部地方在映射下对应着一个平面地图,球面上每个地点对应着平面地图上一个坐标,我们可以用坐标进行这个球面局部的各种计算。用几张平面地图覆盖了全球,就可以计算地球的各处。

对高维和更一般情况,也可以类似地,把拓扑空间X的一个局部开集,一一映射到$ \mathbb{R}^n $ 
空间上来计算。X空间上的一个点x对应着$ \mathbb{R}^n $ 空间上的一个点,称为x的坐标,x的邻域对应着坐标的邻域以保持对应的收敛关系。所以这个映射必须是同胚的,也就是这个一一对应的映射双向都是连续的,就像X中的这个开集通过伸缩变形展平成$ \mathbb{R}^n $ 

空间的开集一样。如果有一族这样的开集覆盖了X,都能做到这样的映射,那么X上的每个点都有了n维实数的局部坐标。这样的X空间便称为流形。覆盖开集的重叠部分,流形上的点在不同映射的局部坐标系上,可以进行坐标变换。因为这样的映射是定义在开集上,所以x点总有一个足够小的邻域是完全在一个映射的局部坐标系上,x点与它坐标的收敛关系是一一对应的,如果交集之处的坐标变换是连续可导的,整个流形通过这些映射的坐标系,便可以有对应的微积分计算,这时称为\textbf{微分流形}。

当然并非任何的拓扑空间都能做到这一点。流形X的拓扑不能太粗,对于两个点必须有能够分开的邻域,即是$ T_2 $或者称为Hausdorff空间;拓扑也不能太复杂,要有可数的拓扑基(其元素的并能够生成所有开集,即是第二可数的)。局部映射必须与相同维数的$ \mathbb{R}^n $  空间同胚。下面是用数学语言描述的定义。

\kaishu\setlength{\leftskip}{1em}

X是第二可数,$ T_2 $的拓扑空间,若在一个覆盖X的开集族中的每个开集,都有一个嵌入$ \mathbb{R}^n $ 的同胚映射,X可以称为\textbf{n维拓扑流形},这个映射称为\textbf{坐标图}。在拓扑流形上,两个坐标图交集部分的点在不同的坐标图上映成不同的(坐标)点,如果这两个坐标变换函数有r阶连续导数,则称它们是$ C^r $相容的坐标图。如果所有坐标图都是$ C^r $相容的,则称这个流形为$ \bm{C^r} $\textbf{微分流形}。r为无穷大时称为\textbf{光滑微分流形}。

\songti\setlength{\leftskip}{0em}

对于一般的距离空间,它是$ T^2 $ ,但只是第一可数的。如果它还是可分的,则它是第二可数的,这个拓扑中任何的开集都能由一组可数开球,用它们的并集来构成。可分的距离空间满足第二可数和$ T^2 $ 的条件,只要每点的开邻域都有同维数的同胚坐标映射,就可以是流形。

两个维数分别为m和n的$ C^r $ 微分流形间的映射称为$ C^r $ 映射,它可以表示为对应点局部坐标上的$ C^r $ 函数。对这个函数的求导和积分,对应着这两个流形间的映射在这局部区域上的相应的运算。比如说,n维光滑微分流形X到R
的函数,在X中点x的邻域对应着$ \mathbb{R}^n $ 

空间上一段光滑曲线。这条光滑曲线,对应着x点的切线(用方向导数表示)是一个n维向量,所有这些切向量形成的空间称为X在x处的切空间。虽然上述的切空间是由某一局部坐标系下定义的,可以证明不同的坐标系导出的切空间是相同的。直观上可以想象成二维X曲面在x这一点上的切平面。如果一个映射F将$ C^r $ 微分流形X上每一点都对应着它切空间上的一个向量,F称为$ C^r $ 向量场,在局部坐标下表示如下,其参数都是$ C^r $ 函数。
\begin{equation}
	F = \sum _{i=1}^n a_i(x)\frac{\partial }{\partial x_i}
\end{equation}

\textbf{纤维丛}的定义了包含三个拓扑空间$ B,M,Y $ 和一个投影映射$ p $ :基空间$ M $ 是全空间$ B $ 的投影 $ p(B)=M $;基空间上每一个点x对应着这个投影在全空间$ B $里的原像$ p^{-1}(x) $,这原像与丛空间Y同胚,称为这点上的丛;基空间上每一点存在着一个邻域$ U $,直积空间$ U\times Y $与$ U $的投影原像$ p^{-1}(U) $同胚。在直观上可以想象二维曲面M,每一点x上都有一根$ p^{-1}(x) $的纤维,这些纤维互不相交,全体构成三维空间$ B $。B中的每一点都可以沿着纤维对应到M的同一个点上(称为投影),全空间上点的邻域在纤维上和投影到基空间上仍然是它们的邻域。不要把基空间$ M $想象成一把刷子的底部,$ M $应该看成是全空间的一个横截面,密实的纤维集束穿过这个横截面向两边无限延伸。每根纤维都像直线Y的弯曲变形。纤维丛的数学模型也可以用来描述物理空间中的场。

微分流形和纤维丛,若以欧几里德三维空间中的曲面和纤维集束几何体来看,都不难想象其图像。不过它们是在抽象的点集拓扑空间上有严格的定义,从而能够在上面推广微积分的应用。这些都是现代微分几何课程的内容,这里的简略介绍,希望通过较精确的数学定义,让大家可以想象这些概念。

【扩展阅读】

\begin{enumerate}
	\item 冯·诺依曼关于量子理论的数学基础,算子环,遍历理论的研究 \url{http://www.kepu.net.cn/gb/basic/szsx/2/25/2_25_1008.htm} 
	
	\item 钱诚德,高等量子力学	\url{http://course.zjnu.cn/huangshihua/book/\%E9\%92\%B1\%E8\%AF\%9A\%E5\%BE\%B7_\%E9\%AB\%98\%E7\%AD\%89\%E9\%87\%8F\%E5\%AD\%90\%E5\%8A\%9B\%E5\%AD\%A6.pdf}
	
	\item 关肇直等,张恭庆,冯德兴,线性泛函分析入门,上海科学技术出版社,1979
	
	\item 维基百科,流形\url{http://zh.wikipedia.org/wiki/\%E6\%B5\%81\%E5\%BD\%A2}
	
	\item 程代展,系统与控制中的近代数学基础,北京:清华大学出版社,2007 \url{http://product.dangdang.com/9350967.html}

\end{enumerate}