\section{重修微积分1——无穷}
这个科普系列是给学过微积分及更深入分析课程的人,觉得读书做题考试都还行,但直观和定义隔了一条河,提到严谨只觉得烦,希望能理解现代分析,又能像物理那样想象的同学。这里介绍无穷、拓扑、空间、测度、泛函和算子等现代分析的概念,让你从高处来看风景。

我学过几次微积分。最初是在高中,了解了导数是变化率,积分能算面积,它们互为逆运算。接着翻到微积分书里的求导公式,一口气背下初等函数导数表和复合函数公式。凭着这个直观概念和套公式本领,大学物理公式推导也看懂了,几天时间里武功大进,高中各种难题,触手可解。这时心雄万夫,认为这就是微积分了。想不明白那巨厚的教材,花那么多的篇幅整什么极限和收敛?后来了解到很多人也和我一样,就学个招式套路,没得心法真传,思想还是停留在有限数学的圈子里,没有走过用逻辑搭起越过无穷深渊的桥,所以永远无法想象不能感知的世界,所说的也只是人云亦云,对与错,自己是无法确认的。再学的实变泛函随机过程,即使考分不错,也只是学个技巧能够解题。真正去应用则在云中雾里,心中懵懂始终不觉踏实。直到在美国数学系从点集拓扑重新学起,每一步踏实了走过去,在巨细的习题中摸过每一块石头,才重新建立起直观想象,找回了自信。

工程师和非数学研究者,不可能都沿着纯数学的路子学习现代严谨的数学理论。耗费大量时间在逻辑巷道里穿行,会让你失去方向看不到全貌,不一定所有人都能走通。对大多数人,数学只是个严谨的逻辑建筑模块和工具。想用好它,最重要的不是繁琐的条件和证明细节的了解,而是要有正确概念的理解。有了正确想象,即使不能精通现代分析,你也能大致读懂相关的理论,用定理套公式才能有自信。急用先学恶补知识时也会有方向。

从初等数学到数学分析,是从现实世界到了无穷空间的跃迁。在牛顿之后,学术界也曾经为此想象图像的转变,迷惑混乱了几百年,至今书里和教授传承还遗留着许多当年混战中的繁杂解读。要走过这座桥,你必须超越有限世界经验的局限,用逻辑证明过的典型例子,来构筑无穷世界的直观想象,知道为什么要分辨看似无谓的概念,接受经验不习惯的事例,才能到达彼岸。

这系列短文,不是教程,而在解读概念。如果你学过这些课程,这里让你依想象指引用逻辑将概念和例子串起来,没学过的也可以浏览无穷世界里的风光。希望有助于迷惘中的人,纠正错误的观念,建立较正确的直观想象,明白些为什么,想穷究深入时,知道往哪儿找。



在微积分之前,数学几乎都是关于“有限”的学问,因为只有是有限的个体,或是它们的有限组合,才算得了,说得清。归纳是人类智力的本能。使用数学归纳法,能够把只对有限事例成立的结论,依仗逻辑和递推关系,严谨地推广到无穷多的适用情况。许多人以为这已是无穷的世界了,其实这只是这个词的一种意思,仍然是在有限的世界里兜圈子。无穷在这里指的是:这个递推过程可以是无法穷尽的。这个认知叫“潜无穷”。在这个无穷尽的过程中,你接触到的每一个还是有限的个体,推理的正确性也只是针对这过程中经过的每一站,无穷的过程没有终点,它们的全体不在考虑之列。

\kaishu
\setlength{\leftskip}{1em}
例如:集合$ \{1\} $是有限集,假如 $ {1,2,\cdots,n} $ 是有限集,那么多一个元素的 $ \{1,2,\cdots,n,n+1\} $ 也是有限集,根据数学归纳法,这对所有n都成立。但这不证明所有自然数的集合$ \{1,2,3 \cdots\} $是个有限集。因为数学归纳法的证明,对所有自然数都成立的命题,是一种潜无穷的陈述,是动态地指对每一个具体的自然数能成立的陈述,而不是指对自然数的全体都成立。

\songti
\setlength{\leftskip}{0em}
有限的度量是实践中可以验证的,潜无穷是以此推理的极限。在物理世界,能够被检验和推测的数量,都是有限精度或组合的;任何计算,都必须在有限的步骤中终止,才有结果;数学证明,必须在有限步骤的推理中完成,才是可信。一句话,现实的世界本来是有限的。人们固有的直观想象,都是在有穷的世界里,用实践经验累积和事例验证而成的。

我们现在普遍认知的连续无限可分世界,是微积分这个数学工具成功应用后,人们在书本灌输下接受的假设。这个看似已经深入人心的观念,对绝大多数人只是个不断被重复的表面陈述,与学前的直观不能融合,难以深思。无穷的彼岸和有限的经验是完全不同的两个世界,归纳法的逻辑无法越过有穷的边界。我们必须先验地承认某些无法验证的观念,用逻辑构造出包囊一切的统一图像,才能消融这个隔阂。

无限的过程能否成为一个数学的量?这早在两千多年前的古希腊,就引起争论,它实际是哲学上的一个观念。无法验证也无关对错。数学是个关心在给定前提下,依照逻辑推理能够走多远的学问。古希腊的毕达哥斯学派,还有个追求的指标是“美”,和谐的美、统一的美、简洁的美。主张无穷不仅仅是个变动的过程,而且这无限过程,代表着它要达到的数学实体,可以用来参与计算,称之为“实无穷”,是把它作为一种数学完备化的扩张。

认为无穷不仅仅是一个过程,而且代表着一个要达到的数值,这在逻辑上很重要。只有这样我们才可以把它放在等式里。比如说,认为无穷循环的 $ 0.999... $ 是数x,才有$ 10x=9.999\cdots $,因此$ 10x = 9 + x $,得出$ x = 1 $. 阿基米德用了同样的原理,将阿基里斯追逐乌龟的过程,写成无穷级数的和$ 1+0.1+0.01+0.001+\cdots $,把它当作一个数x,由$ 10x=10+x $关系,才能算出他追上乌龟在x=10/9处。牛顿和莱布尼茨定义导数为两个无穷小量dy和dx之比dy/dx,到了后来柯西修正为导数是Δy/Δx分子分母都趋于0的无穷过程。这些都只有承认实无穷观点时,才有可能。

这些都是假定实无穷能够保持算术运算的和谐关系,才得出答案的。历史上数学天才运用这个思想,取得了令人惊异的成绩。但是这个扩张并不总是像引入开平方的无理数那样和谐。1703年,意大利数学家格蓝迪问大家,无限的过程$ 1-1+1-1+1-1+\cdots $,等于多少?数学王子欧拉,用阿基米德相同的方法,假设$ x=1-1+1-1+1-1+\cdots $,将这个无限过程第一项1先拎出来,后面剩下的相当于它自己乘上-1,有了$  x=1-x $,欧拉得出格蓝迪级数的和是1/2。但是有人说,根据加法的结合律,有$ x=(1-1)+(1-1)+(1-1)+\cdots=0+0+0+\cdots=0 $,类似的做法还可以有$ x=1+(-1+1)+(-1+1)+(-1+1)+\cdots=1+0+0+0+\cdots=1 $。这格蓝迪级数到底是0、1/2还是1?贝克莱主教质问牛顿:你这个无穷小在做除法时认为不是0,在做加减法时当着是0,它到底是0还是个非0的数?

其实古人早已疑惑:自然数全体的数量是无穷大,在其中拿掉几个数,它还是一样大吗?无穷循环小数 $ 0.999... $的每一个截断都小于1,它怎么最终不再小于1?承认和不承认实无穷的存在,都与某一种数学的美相冲突。这就是数学家在历史上,对实无穷和潜无穷无所适从的原因。

在牛顿之前,数学家对实无穷采取回避的态度,除了个别公认的天才,有底气做别人不敢质疑的猜测,大家都只老老实实地把无穷只看是一个过程,这样心里踏实,这也是潜无穷派的想法。但微积分是关于无穷的算术。到这时已是无可回避了。

这些冲突,实际上是企图将本质不同东西纳入过去直观的框架。对于有穷的东西,我们可以验证它,从已知的数学实体中通过有限步的推理确认它。在实无穷与确定的数之间,却没有能够确信的桥梁。无穷的序列,可能等价于已知的数学元素,也可以不是大家熟悉的东西,也可能是非此非彼的不确定。把本质上不能一致定义的东西,放入有限数学通行的等式里,从不同的方向则会推出不同的东西。这便是发生矛盾的原因。

现代数学基本认可实无穷的观念,但不认为它的所指都有意义。这个无限的过程,并非都能代表一个数。有的不是指向大家熟悉的数学元素;有的并没有确定的数学意义。实无穷的提法,因为涉及过多历史上混乱的争议,在数学界也不再强调了,而着眼于具体的处理。

将实践中无法检验的一种概念作为构筑理论的砖块,在思想史上并不罕见。在哲学、宗教、社会科学、甚至在自然科学理论里比比皆是。人们希望用新概念的理论,能够透视现实中的秘密,用之推演的结果,与我们的观察有某种程度的吻合。数学家的任务是将它精确化,限定适用的范围,以保证在逻辑上不至于造成冲突。科学是用逻辑为混沌的自然立法,让世界看起来比较有条理。

集合论作为现代数学的基石,研究了无穷大这个新的数学实体。它用一一映射的关系,回答了无穷大比较的问题。这个无穷大与有限的数,只保持一种序的关系,它们是不在一个层次数量,有限数的运算不完全适用于它。这提醒人们,不能用处理有限世界的那种直观,来看待用逻辑扩张出来的无穷世界。要自信自如地在这世界里玩,你必须理解新的概念,接受一些不习惯的事例,用逻辑推演出来的事例来纠正旧的想象,形成新的直观图像。

用无穷变化的过程代表一个可以达到的数,这需要另一座桥梁,我们必须知道这是在什么意义下的等价关系,在什么条件下才有可能,这需要理解一个概念——收敛。