\chapter{重修微积分5——线性}

数学的空间是集合上赋有某些数学性质的论域。距离空间赋予集合中点的远近概念,让我们可以很直观地想象无穷空间中的收敛、极限和连续。微分、积分和求极限的计算都是线性的,线性运算依赖于,所在空间的点有着对线性运算封闭的代数结构。微积分中大部分概念,可以在有拓扑结构和线性结构的点集空间中理解和推广。所以我们要了解同时拥有这两种结构的空间。

集合X中的元素如果对线性运算封闭,即对于数域(实数或复数)$ \mathbb{K} $
\[x,y \in X, a \in \mathbb{K} \rightarrow x+y \in X, ax \in X \]

那它是线性空间,其中元素叫向量。赋予向量``长度''(范数)的概念,则可以导出距离。

\kaishu\setlength{\leftskip}{1em}

设X是数域K上的线性空间,X上的范数定义为从X到R一个映射:$ x\mapsto \Vert x\Vert $ ,它对于任意的$ x,y\in X,a\in\mathbb{K} $ 满足非负性,比例性和三角不等式:
\[\Vert x\Vert\ge 0, \Vert x\Vert =0 \Leftrightarrow x=0\]
\[\Vert ax\Vert=|a|\Vert x\Vert\]
\[\Vert x+y \Vert  \le \Vert x\Vert + \Vert y\Vert\]

$ \Vert x\Vert $叫做点x的\textbf{范数}(norm),赋以范数的线性空间叫做\textbf{线性赋范空间}(Normed linear space)或简称\textbf{赋范空间},记为$ (X,\Vert \cdot\Vert) $.

\songti\setlength{\leftskip}{0em}

线性赋范空间是定义了``长度''的线性空间,空间中两个点之差也是个空间中的点,以此可用长度来定义距离。令 $ d(x,y)=\Vert x−y \Vert $,则定义了一个距离。如果线性赋范空间按其导出的距离是个完备的距离空间,则称它是巴拿赫(Banach)空间 。线性空间中的点有时也称为向量,以强调它的线性元素性质。

在线性代数中,大家熟悉的是有限维的线性空间,赋向量予范数,当然也成了赋范空间。所以实数以其绝对值为范数,n维欧几里德空间以其分量平分和的开平方为范数,都是巴拿赫空间。显然还可以以其他方式来定义不同的范数,可以证明n维赋范空间不同定义的范数都是等价的,因此n维巴拿赫空间不同定义的距离也都是等价的。就是说,它们在一种定义下收敛,在另一种定义下也必然如此。从实数的完备性,不难推出它们的完备性。这些已是大家熟知的结果。

对巴拿赫空间,人们更感兴趣的是无穷维的情况,它的性质略有些不同。让我们看几个例子。

\kaishu\setlength{\leftskip}{1em}

例5.1:在闭区间$ [a,b] $上所有连续函数的集合$ C[a,b] $,对任意的$ f\in [a,b] $ 定义范数 $ \Vert f \Vert = \max_{a\leq t \leq b} |f(t)| $,它是一个巴拿赫空间。

\songti\setlength{\leftskip}{0em}

线性空间的维数是其中最多线性不相关向量的个数。在闭区间$ [a,b] $上的多项式是这个巴拿赫空间上的向量,它们不可能都是有限个幂函数$ x^n $的线性组合,所以这空间是无穷维的。

\kaishu\setlength{\leftskip}{1em}

例5.2:记无穷数列$ x = (x_n) $为集合中的一个点,对于任何一个自然数$ p $,都可以定义一个范数 $ \Vert x \Vert_p = (\sum_{n=1}^\infty |x_n|^p)^{1/p},\;\;\; 1\le p<\infty  $

对于给定的自然数$ p $,范数值有限的无穷数列的集合,在这范数下是一个巴拿赫空间,这空间很有用,记为$ l^p $ 。

例5.3:在$ E $上$ p $次勒贝格可积函数,在下面范数下是一个巴拿赫空间,这是分析中更常遇见的空间,记为$ L^p(E) $
, $ \Vert f \Vert_p = (\int_E|f(x)|^p dx)^{1/p},\;\;\; 1\le p<\infty $

\songti\setlength{\leftskip}{0em}

下面将看到以$ L^2(E) $最为著名。$ l^\infty,L^\infty(E) $ 及更多巴拿赫空间的例子见【1】。

既然线性空间的点,被看作是向量,从几何直观上很自然会想象到夹角的概念,如果用正交的向量做分解,在物理和工程上会很有用。这要推广欧几里德空间的内积概念。

\kaishu\setlength{\leftskip}{1em}

设X是数域$ \mathbb{K} $(通常指实数或复数)上的线性空间,定义映射$ \left\langle x,y \right \rangle $ :$ X\times X\rightarrow \mathbb{K} $,称为X上的内积,如果它满足下列性质:$ \forall x,y,z\in X, \forall a,b \in \mathbb{K} $ 有

正定性:$ \left\langle x,x\right \rangle \ge 0, \left \langle x,x \right \rangle = 0\Leftrightarrow x=0 $;

共轭对称性:$ \left\langle x,y \right \rangle = \overline{\left \langle y,x \right \rangle} $;

对第一变量的线性:$ \left\langle ax+by,z \right \rangle = a\left \langle x,z \right \rangle + b \left\langle y,z \right \rangle $ 

赋以内积性质的线性空间称为内积空间。(注:有的教材的定义是对第二变量线性。)

定义范数$ \Vert x \Vert = \sqrt{\left \langle x,x \right \rangle} $ ,如果按其导出的距离是个完备的距离空间,则称它是\textbf{希尔伯特(Hilbert)空间}。

\songti\setlength{\leftskip}{0em}

不难看到,例5.2和5.3中$ l^2 $和$ L^2 $空间,很容易用内积来定义。

\kaishu\setlength{\leftskip}{1em}

例5.4:$ l^p $空间中无穷数列$ x =(x_n), y = (y_n) $定义它们的内积 $ \left \langle x,y \right \rangle  = \sum_{n=1}^\infty x_n \overline{y_n} $

不难验证,以此导出的正是前面$ l^2 $ 空间定义的范数,所以$ l^2 $ 空间是希尔伯特空间。 $ L^2 $也可以仿此定义内积, $ \left \langle x,y \right \rangle  = (\int_E x(t) \overline{y(t)} dt)^{1/2} $,所以它也是希尔伯特空间【2】。

\songti\setlength{\leftskip}{0em}

学习线性代数时,我们知道,任何向量都可以表示为在线性空间下,基向量的线性组合。对巴拿赫空间和希尔伯特空间中的点,比如说连续函数、可积函数,是不是也可以做到这一点?当然如此,不过这个线性组合很可能是无穷级数的和或积分。比如说例5.1中连续函数可以表示为多项式级数的和(斯通-维尔斯特拉斯定理:闭区间上多项式一致逼近连续函数)。

在希尔伯特空间,因为有了内积概念,我们可以有正交归一的基,即对于它们有 $ \left \langle e_i,e_j \right \rangle $$= \delta_{ij} $
。例如$ e_i $为无穷数列中第i个分量为1,其余分量都为0的数列,集合$ \{e_k| k \in \mathbb{N}\} $ 是$ l^p $空间上的基,而且是 $ l^2 $空间上的正交归一基。

在数学物理和工程应用中,为了便于微分方程求解,人们经常将函数分解为某种无穷函数项级数的和,如傅立叶级数、各种正交多项式、特殊函数等等。这样可以把微分方程变成代数方程来求解。这其实是将线性空间中的向量表示成在一组基上的线性组合。这个函数是否等于这种分解,实际上是问这个无穷级数是否收敛。学习了拓扑概念后,我们就会问这是哪一种意义下的收敛。最简单的如幂级数等指的是逐点的收敛,这只是说在收敛域里,每个点的函数值满足收敛关系,极限函数不一定保持序列中函数的性质,如有界性,连续性,可积性等等。有时我们更关心的是,能保持有某些性质的分解表示问题,这就需要了解所在的拓扑空间。

无穷维的巴拿赫空间和希尔伯特空间中的向量,可以分解为对基向量的线性组合,这并不意味着可以表示成无穷级数的和,因为基不一定是可数的。例如 $ e^{-st},\;\;s\ge 0 $ 是一族函数集合的基,它是不可数的。能否做到对可数基的分解,取决于空间的拓扑,或是怎么定义它的范数和内积。

巴拿赫空间有可数的基(向量),当且仅当它是\textbf{可分的},也就是说这空间有可数的稠集。显然,如果希尔伯特空间的向量都能表示成正交归一基的无穷级数和,它是可分的。反之,可分的希尔伯特空间中的向量,都能表示成正交归一基的无穷级数和。

对于可分的希尔伯特空间,将一个可数的稠集的点排成序列,剔去与序列中前面线性相关的点,形成了一个线性独立的序列,再将这组序列用Gram-Schmidt方法正交归一化,都可以形成了一个可数的正交归一基($ e_k $)。可分的希尔伯特空间中的点x,都可以在这个基上分解,也就是等于收敛的级数。 $ x = \sum _{k=0}^\infty \langle x, e_k \rangle e_k $

也就是说在基($ e_k $)下,x一一对应着数列 $ \langle x, e_k \rangle $,不难证明这个数列是$ l^2 $空间的点,这个对应保持了线性关系,极限和内积的不变,即它与$ l^2 $ 空间等价。可分的希尔伯特空间在内积的对应关系下是等价的。

尽管可分的希尔伯特空间在内积的抽象下可以看成是唯一的。空间中的点都可以等于一个收敛的级数。但是不同的内积和不同线性独立向量的选取,可以形成不同的正交归一基,方便用于不同的应用。比如说应用于通讯工程信号处理的$ L^2(0,1) $中的Haar函数组与Walsh函数组,物理中用到的Hermite多项式等。

不能表示为无穷级数和的函数,也可能表示为不可数基向量的线性组合,即积分变换。比如说,对不可数基$ e^{−ist} $ 向量用内积公式来分解的傅立叶变换,和对基向量线性组合(积分)的傅立叶变换反演。

学习数学概念,最起码的功课是用一个简单的例子,根据定义自己走过一遍,才能得到真正的体会。如果你相信自己是理解了,例5.2,5.3和5.4中$ l^2 $和$ L^2 $是两个简单的例子。请从内积、范数、距离、收敛、完备的概念开始,验证它们符合定义,证明它们是巴拿赫空间,并且也是希尔伯特空间,而且是可分的。这些证明都不需要技巧,范数和内积中的不等式可以直接引用Minkowski和Holder不等式【4】【5】,其他都没有难度,只是验证对概念的理解。

【扩展阅读】
\begin{enumerate}
	\item 互动百科,巴拿赫空间 \url{http://www.baike.com/wiki/\%E5\%B7\%B4\%E6\%8B\%BF\%E8\%B5\%AB\%E7\%A9\%BA\%E9\%97\%B4} 
	
	\item 维基百科,希尔伯特空间\url{http://zh.wikipedia.org/wiki/\%E5\%B8\%8C\%E5\%B0\%94\%E4\%BC\%AF\%E7\%89\%B9\%E7\%A9\%BA\%E9\%97\%B4}
	
	\item 程代展,系统与控制中的近代数学基础,北京:清华大学出版社,2007 \url{http://product.dangdang.com/9350967.html}
	
	\item 维基百科,闵可夫斯基不等式\url{http://zh.wikipedia.org/wiki/\%E9\%97\%B5\%E5\%8F\%AF\%E5\%A4\%AB\%E6\%96\%AF\%E5\%9F\%BA\%E4\%B8\%8D\%E7\%AD\%89\%E5\%BC\%8F}
	
	\item 维基百科,赫尔德不等式\url{http://zh.wikipedia.org/wiki/\%E8\%B5\%AB\%E5\%B0\%94\%E5\%BE\%B7\%E4\%B8\%8D\%E7\%AD\%89\%E5\%BC\%8F}
\end{enumerate}




