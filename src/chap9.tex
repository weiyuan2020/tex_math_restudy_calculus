\chapter{重修微积分9---泛函}

算术是从一个或几个具体的数计算另一个数的学问,在这儿,数是已知和不变的。代数等式中有些数虽然未知,但其所指,仍是一个固定的数。在几百年前,只在几何中表示数量的对应,运动则代表着变化。几千年中的算术、几何和运动的研究,人们在三者间交叉借用形象来类比推理,直到1694年莱布尼茨终于用了函数这名词,抽象地表达变动的已知数到答案数之间,算法所对应的映射。其后近百年间,由约翰·伯努里和他的学生欧拉的推崇,最后到维尔斯特拉斯,确认了必须用函数的概念,把微积分建立在代数而不是几何的基础上。

函数,是我们从小学算术到中学要理解的第一个抽象概念,没过这坎的人,与数理绝缘,后面的课就不能理解了,对数学的认知停留在中世纪。初等微积分是在函数概念的基础上,对变动数极限运算的数学,牛顿称之为``流数术'',在有穷的世界窥测无穷的彼岸。近代分析建立在无穷空间的映射概念上,研究抽象空间结构和算子性质。由此俯视分析理论,能更抽象地构造数学模型,解决微分方程解和函数推广等等难题。理科生在这里,必须再过一个坎,走进抽象无穷的世界,才能理解现在的数学,而不是停留在二百年前的旧时光里。

算术的眼界局限在数域中。函数表达了数域中变量与映射值的对应关系。经典微积分用函数和极限的概念从数域,跨进实数和欧几里德空间。其运算都基于这个空间的性质。泛函分析将函数作为变量,研究它所在的空间和算子。在这里,一个函数也只看成集合上的一个点。

将讨论的对象抽象成集合中的点,点与点之间的相邻关系和点间运算对应关系,是集合上设定了的性质。数学的空间是定义有这些性质的集合,在这些设定条件下来讨论数学问题,而不再借助任何其他背景。在我们介绍过的空间里,由粗到精的包含关系顺序是:拓扑空间,$ T^2 $空间,距离空间,赋范空间,巴拿赫空间,希尔伯特空间。这些空间都只是抽象的类,可在相应的各类里设定具体的拓扑、距离、范数或内积。$ L^2 $和$ l^2 $空间,欧几里德空间,实数空间则是常见具体化的希尔伯特空间。初等微积分局限在实数空间和欧几里德空间里,泛函分析研究抽象的空间,特别是赋范空间、巴拿赫空间和希尔伯特空间的结构和线性运算性质。下面带你领略这里的风光。

两个距离空间中的映射称为算子。这篇只讨论赋范空间的线性算子。回顾一下赋范空间定义,它是线性空间,是以向量长度为范数导出了距离的距离空间。在这距离定义下,如果它对收敛还是完备的,则称为巴拿赫空间(Banach space)。大家熟悉的欧几里德空间$\mathbb{R}^n$,是有穷维的巴拿赫空间,其线性算子在基底下表示为矩阵。无穷维巴拿赫空间中线性算子的研究,是泛函分析的中心内容。

在分析中,函数是数与数的对应关系,是实数或复数间的映射,泛函则是以函数为自变量,对应于实数或复数值的映射。一般地说,从距离空间到数域的映射称为泛函。数域也是赋范空间,所以线性泛函也是一种线性算子。

\kaishu\setlength{\leftskip}{1em}

例9.1:函数的定积分是个线性泛函。下面$ L(f) $和$ K(f) $都定义了$ L^1[0,1] $空间上的一个泛函。(在[0,1]上绝对可积函数的空间上)

\songti\setlength{\leftskip}{0em}
\begin{equation}
	L(f)= \int_0^1f(t)dt \;\;\; \forall f \in L^1[0,1] \;\;\;\;\;\;
	K(f)= \int_0^1f(t)e^{-t}dt \;\;\; \forall f \in L^1[0,1]
\end{equation}

先介绍线性泛函的一些性质,以此来揭示赋范空间、巴拿赫空间和希尔伯特空间的结构。

对线性算子T,如果存在着一个正数c,对其定义域上所有的点都有$ \|Tx\| \leq c\|x\| $,称这个算子是有界的,这个c的下确界称为\textbf{线性算子T的范数}。对于线性算子,连续性与有界性是等价的。有界的算子总是把微小的变化映射成微小的差异,把有界的集合映射成有界的像。

泛函的连续性在应用上很重要,例如用一个收敛的函数序列来计算泛函作用下极限值,只有对连续泛函这样的逼近才有意义。欧几里德空间的线性泛函,可以表示成一个内积,它总是连续和有界的。但在赋范空间,并非所有的线性泛函都是有界或连续的。

\kaishu\setlength{\leftskip}{1em}

例9.2:闭区间[0,1]上连续可微函数集合$ C^1[0,1] $,以$ \|x\| = \max_{0\leq t \leq 1} \|x(t)\| $ 为范数构成赋范空间。函数在0点的导数是这空间的一个线性泛函。它不是连续的。因为对函数序列$ x_n(t)=\frac{1}{n}\sin(nt),n\geq 1 $,有$ \|x_n\| = \frac{1}{n} \rightarrow 0, \; n \rightarrow \infty $,即$ x_n \rightarrow 0 $,但是$ {x_n}'=\cos(n0)=1, \; n \geq 1 $ ,它并不趋于0。所以它不是连续的,也不是有界的。

\songti\setlength{\leftskip}{0em}

在某些线性距离空间,甚至没有非零的有界线性泛函。但是对赋范空间,我们却有足够多的有界线性泛函。

\textbf{Hahn-Banach延拓定理}:如果赋范空间的线性子空间上,定义有一个有界线性泛函,那么可以把它延拓到全空间,延拓后的算子也是个有界线性泛函,在原来子空间的映射保持不变,而且它的范数与延拓前是一样的。

取X中任何一个非零点$ x_0 $,它的数乘张成一维的线性子空间,在这子空间上定义一个有界线性泛函,使得$ f(x_0)= \|x_0\| $ ,应用这个定理,可以将它延拓到全空间,并且有$ \|f\| =1 $,这说明对于任何一个赋范空间,都有不比它向量少的有界线性泛函。

记赋范空间X上所有的有界线性泛函的集合为$ X^* $,不难验证$ X^* $在算子的范数下是一个巴拿赫空间。$ X^* $叫做X的\textbf{对偶空间},也称为\textbf{共轭空间}。$ X^* $的对偶空间$ X^{**} $,自然也是个巴拿赫空间。那么$ X^{**} $与X是什么关系?

对于X上的点x,可以定义$ X^* $上的泛函:$ L_x(f) = \overline{f(x)},\;\;  \forall f \in X^* $ 

显然,$ Lx $是线性的,而且$ \|Lx\| = \|x\| $,是有界的。这说明X到$ X^{**} $间有个一一的,线性的,并且保持范数相等的映射,即X等价于$ X^{**} $的一个线性子空间。如果这个映射还是满的,即X等价于$ X^{**} $,则称为X是\textbf{自反的},记为X=$ X^{**} $,自反的赋范空间必定是个巴拿赫空间。


\kaishu\setlength{\leftskip}{1em}

例9.3:函数空间$ L^p[0,1], p>1 $是自反的,它上面的线性泛函$ f(\cdot) $ 表示为
\[f(x)=\int_0^1 x(t)\overline{y(t)}dt, \;\;\; x\in L^p[0,1],\; y \in L^q[0,1], \;\frac{1}{p}+\frac{1}{q}=1\]

它的对偶空间是$ L^q[0,1] \;\; \frac{1}{p}+\frac{1}{q}=1 $

\songti\setlength{\leftskip}{0em}


自反巴拿赫空间与对偶空间互为有界线性泛函的关系,让我们联想起内积的关系。所以就用内积的符号来表示线性泛函,$f(x)=\langle x,f\rangle$
,对它们间的线性性质与内积形式上完全一样,只不过这里左右矢量是在不同的空间。特别地,从线性算子范数的定义,有与内积完全相同的Schwarz不等式$  |\langle x,y\rangle |\leq \|x\|\|y\| $

希尔伯特空间H是定义了内积,并以此导出范数的巴拿赫空间。由上面巴拿赫空间与对偶空间的内积表示,及Schwarz不等式,很自然地会猜测:H空间的对偶空间是否是它自己?确实如此。H中任何一点,都可以用内积定义H空间上的一个有界线性泛函,这说明H是H*的子集。\textbf{Riesz表现定理}则证明,H空间上,任何一个有界线性泛函$ f\in H^* $,都对应着空间中的一个点$ y\in H $,使得$ f(x)= \langle x,f\rangle ,\forall x\in H $,而且$ \|f\| = \|y\| $,这说明$ H^* $是$ H $的子集。所以$ H=H^* $。

Hahn-Banach延拓定理证明了每一个巴拿赫空间,它的有界线性泛函构成了它的对偶的巴拿赫空间,有界线性泛函算子间的作用可以用内积的式子来表示。Riesz表现定理则肯定了希尔伯特空间的对偶空间就是它自己。Hahn-Banach延拓定理可以放宽到,具有线性的距离空间附加上一些条件,泛函分析的教科书介绍这方面的内容。

距离空间中收敛的要求比较强,用泛函我们可以定义一种比较弱的``功能性''的收敛。

比如说,赋范空间X中的序列$ (x_n) $收敛于$ x_0 $,指 $ \lim_{n\rightarrow \infty}\| x_n - x_0 \| $

这有时称为\textbf{强收敛},\textbf{弱收敛}则定义为这序列对所有的有界线性泛函都有

\[\lim_{n\rightarrow\infty}f(x_n) = f(x_0),\;\; \forall f \in X^*\]

$ X^* $中序列$ (f_n) $ \textbf{弱*收敛}于$ f_0 $,则是满足

\[\lim_{n\rightarrow\infty}f_n(x) = f_0(x),\;\; \forall x \in X\]

显然强收敛隐含着弱收敛,弱收敛未必能强收敛,下面是个例子。

\kaishu\setlength{\leftskip}{1em}

例9.4:希尔伯特H的任何正交归一基{ $ e_n $ },不难从向量在这个基上分解的无穷序列和中得到,$ \lim_{n\rightarrow\infty}\langle x,e_n \rangle =0,\;\forall x\in H $

由Riesz表现定理得知,这表明这基的序列弱收敛于0,但是所有基向量的范数都是1,所以它不可能强收敛于0.

\songti\setlength{\leftskip}{0em}


【扩展阅读】

\begin{enumerate}
	\item 关肇直等,张恭庆,冯德兴,线性泛函分析入门,上海科学技术出版社,1979
	
	\item 程代展,系统与控制中的近代数学基础,北京:清华大学出版社,2007 \url{http://product.dangdang.com/9350967.html}
	

\end{enumerate}


	转载本文请联系原作者获取授权,同时请注明本文来自应行仁科学网博客。
链接地址:\url{http://blog.sciencenet.cn/blog-826653-892196.html }